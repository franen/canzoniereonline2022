%indicazione per la compilazione:
% 1. compila con LuaLaTeX
% 2. compila con LuaLaTeX
% 3. apri il prompt dei comandi nella cartella di lavoro e dai il comando xindy -M texindy -C utf8 -L italian -M xindystyle alfabetico.idx
% 4. compila con LuaLaTeX

\documentclass[standard,authorsindex,titleindex,tematicindex]{canzoniereonline}
\usepackage{fontspec}
\usepackage[italian]{babel}
\defaultfontfeatures{Ligatures=TeX}

\geometry{paperheight=148mm, paperwidth=210mm, inner=8mm, outer=10mm, top=10mm, bottom=10mm}%scrivi qui dimensioni pagine e margini

\usepackage[]{microtype}

\usepackage[bookmarks]{hyperref}


\begin{document}
\thispagestyle{empty}
%\includepdf[]{copertina}
%\includepdf[pages=1]{cover_miotto_v1} %metti qui il nome del pdf della copertina
\cleardoublepage
\vspace*{20mm}
\hskip100mm\begin{minipage}{50mm}\LARGE \scshape
canzoni 

liturgiche 

scout 

e tante altre\dots
\vspace*{50mm}
\end{minipage}

%\newpage
\thispagestyle{empty}
{\small
\null\vfill

\begin{minipage}{0.5\textwidth}
\setlength{\parindent}{0pt}
\vfill
Questo libretto è stato composto con \LaTeX{} utilizzando il pacchetto \textsf{songs}\footnote{\url{http://songs.sourceforge.net}} di Kevin W. Hamlen. 
È distribuito con licenza Creative Commons.
\bigskip

Il font utilizzato è il \textsf{Libertine}\footnote{\url{www.linuxlibertine.org}}.
\bigskip

\textit{Impaginazione}:\\
pinco pallo
\bigskip


Per ogni suggerimento, correzione o richiesta del sorgente potete scrivere a:\\ \texttt{\footnotesize pincopallo at gmail dot com}
\bigskip

I testi delle canzoni sono di proprietà dei rispettivi autori.
\bigskip

Finito di impaginare il \today.
\bigskip
%\byncsa

{\hspace*{3.3em}\Large\ccbyncsaeu}
\end{minipage}

\cleardoublepage



\begin{songs}{}

%METTI QUI TUTTE LE CANZONI

%titolo{A Betlemme di Giudea}
%autore{Tradizionale Francese}
%album{}
%tonalita{Re}
%famiglia{Liturgica}
%gruppo{}
%momenti{Natale}
%identificatore{a_betlemme_di_giudea}
%data_revisione{2011_12_31}
%trascrittore{Francesco Endrici}
%video{https://www.youtube.com/watch?v=l1Okqh9nxfA}
\beginsong{A Betlemme di Giudea}[by={}]
\beginverse
\[D]A Betlemme \[A]di Giu\[D]{dea} \[D]una gran luce \[A7]si le\[D]vò:
\[B-]nella notte, \[A]sui pa\[D]stori, \[B-]scese l'annuncio e \[A]si can\[D]tò.
\endverse



\beginchorus
\[D]\[B-]\[E-]\[A7]\[D]\[G]Glo\[A]{ria} \[D]in ex\[G]celsis \[D]{De}\[A]o
\[D]\[B-]\[E-]\[A7]\[D]\[G]Glo\[A]{ria} \[D]in ex\[G]celsis \[D]\[A]De\[D]o
\endchorus

\beginverse
\chordsoff
Cristo nasce sulla paglia, \brk figlio del Padre Dio con noi,
Verbo eterno, Re di pace, \brk pone la tenda in mezzo ai suoi.
\endverse

\beginverse
\chordsoff
Tornerà nella sua gloria, \brk quando quel giorno arriverà:
se lo accogli nel tuo cuore \brk tutto il suo Regno ti darà.
\endverse
\endsong

%titolo{Al passo del guidon}
%autore{A. Mazzocolin, E. Demattè}
%album{}
%tonalita{Do}
%famiglia{Scout}
%gruppo{}
%momenti{}
%identificatore{al_passo_del_guidon}
%data_revisione{2013_01_16}
%trascrittore{Francesco Endrici}
\beginsong{Al passo del guidon}[by={Mazzocolin, Demattè}]
\beginverse
Al \[C]passo del guidon, \brk fratello scout t'attende l'avven\[G]tura
tra il \[D-]verde delle macchie e sotto il \[G]{sol\dots}
Al \[C]passo del guidon, \brk avanti ad esplorare la na\[G]tura:
un \[D-]nido, un'erba, un fior t'aspetta ed \[G]è \brk tut\[C]to per \[G]te.
\endverse
\beginchorus
\[C]A\[G]pri l'\[C]occhio fratello scout,
tutto il \[G]mondo che è intorno a te 
è una \[A-]cosa meravi\[D-]glio\[G]sa.
\brk
\[C]A\[G]pri l'\[C]occhio fratello scout,
tutto il \[G]mondo che è intorno a te 
è una \[A-]cosa meravi\[D-]gliosa \brk \[G]da sco\[C]prir. \[G]\[C]
\endchorus
\chordsoff
\beginverse
Al ^lato del sentier, \brk la pista ancor, fratel, non è bat^tuta
la ^bussola ti guida senza er^{ror\dots}
Al ^lato del sentier \brk il mondo è tutta terra scono^sciuta:
ma ^certo c'è un amico che di ^là ti a^spette^rà.
\endverse
\beginverse
Al ^fuoco del falò \brk la gioia dei fratelli è la più ^pura 
fa un ^unica gran tenda il vasto ^{ciel\dots}
Al ^fuoco del falò \brk si sente ancor più limpida e si^cura
la ^voce che ci vuole esplora^tor sul ^nostro o^nor!
\endverse
\endsong

\end{songs}


\printindex[alfabetico]


\end{document}
